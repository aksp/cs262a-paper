\section{PeerDB implementation}

PeerDB is implemented as a library for Meteor web framework.
It provides a declarative way to define database schema for documents and relations between document for your Meteor web application.
Together with specifying relations, you can also specify which fields should be embedded as subdocuments for both forward and backwards relations.
Currently, PeerDB requires programmer to decide if and which fields to embed.
That decision can often be made on program's logic and profiling of queries, in a similar way one would be deciding about existence of an index on a field.
Both embedding and indexes make a trade-off between faster read times on an expense of write times.
In later sections we analyze this trade-off and present some general guidelines and automatic algorithm to help deciding if you want to embed fields or not.

Additionally, PeerDB provides an easy way to define generator fields (fields which value is computed based on values of other fields) and generic triggers.

It uses an abstraction over MongoDB oplog to implement all above mentioned features.
MongoDB oplog originally serves for replication among multiple MongoDB instances, sending a stream of all changes from master to slaves.
We can connect to this same oplog to observe all changes in the database and determine if a change is connected to a field which is embedded in a related document.
If this is so, PeerDB issues update queries which update all those fields in subdocuments in related documents.

As currently implemented, PeerDB issues updates queries in a straightforward way, without any optimizations which might reduce the number of unnecessary updates queries.
Currently queries are issued and are left to MongoDB to determine that a particular query does not have anything to update.

PeerDB runs as a background process inside Meteor application, observing changes in the database and issuing the updates.
To scale, it provides a way to run as multiple separate processes/instances to distribute the load, each instance observing and reacting to just a subset of all documents based on their ID.

It is implemented in CoffeeScript and available as open source library at \url{https://github.com/peerlibrary/meteor-peerdb}.