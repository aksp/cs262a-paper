Simple document-oriented databases like MongoDB forgo many traditional features of relational databases in favor of better performance. 
This means that applications often need to build additional features on top of the simple provided features. 
For example, applications often have to resolve relationships between stored documents. 
Recursive queries incurred by resolving relationships introduce delays. 
To mitigate such delays, we developed PeerDB, a modification to MongoDB. 
PeerDB optimizes for the common use case where the main document only requires a subset of fields from any related documents. 
PeerDB stores subsets of related fields as subdocuments of the main document. 
In theory, this modification should reduce existing recursive read delays as applications will no longer need to recursively query documents in many cases. 
However, we have not quantitatively evaluated PeerDB to find out whether it improves performance of MongoDB. In addition, there is no method for automatically selecting which fields to embed using PeerDB.

We contribute a comparison of read and write times between three system: PeerDB, MongoDB and PostgreSQL. 
In addition, we make each comparison with both low-level queries and high-level web applications. 
We find that although PeerDB contributes to faster reads than other systems under some conditions, this performance gain comes with trade-offs (e.g. write time and traffic). 
To help developers manage these trade-offs when choosing which fields to embed, we propose and evaluate a new algorithm for automatically selecting which fields to embed based on a given read and write workload. 
Our algorithm compares many possible configurations using a novel cost model to predict expected impact, then returns the lowest cost configuration.
We test our algorithm under different workloads and parameter settings.