\documentclass{sigchi}

% Use this command to override the default ACM copyright statement (e.g. for preprints).
% Consult the conference website for the camera-ready copyright statement.

%% EXAMPLE BEGIN -- HOW TO OVERRIDE THE DEFAULT COPYRIGHT STRIP -- (July 22, 2013 - Paul Baumann)
\clubpenalty=10000 
\widowpenalty = 10000
%% EXAMPLE END -- HOW TO OVERRIDE THE DEFAULT COPYRIGHT STRIP -- (July 22, 2013 - Paul Baumann)
\toappear{CS262A Class Project}

% Arabic page numbers for submission.
% Remove this line to eliminate page numbers for the camera ready copy
% \pagenumbering{arabic}


% Load basic packages
\usepackage{balance}  % to better equalize the last page
\usepackage{graphics} % for EPS, load graphicx instead
\usepackage{times}    % comment if you want LaTeX's default font
\usepackage{url}      % llt: nicely formatted URLs
\usepackage{color}    % now you can define colors
\usepackage{array}
\newcolumntype{L}[1]{>{\raggedright\let\newline\\\arraybackslash\hspace{0pt}}m{#1}}
\newcolumntype{C}[1]{>{\centering\let\newline\\\arraybackslash\hspace{0pt}}m{#1}}
\newcolumntype{R}[1]{>{\raggedleft\let\newline\\\arraybackslash\hspace{0pt}}m{#1}}


% llt: Define a global style for URLs, rather that the default one
\makeatletter
\def\url@leostyle{%
  \@ifundefined{selectfont}{\def\UrlFont{\sf}}{\def\UrlFont{\small\bf\ttfamily}}}
\makeatother
\urlstyle{leo}


% To make various LaTeX processors do the right thing with page size.
\def\pprw{8.5in}
\def\pprh{11in}
\special{papersize=\pprw,\pprh}
\setlength{\paperwidth}{\pprw}
\setlength{\paperheight}{\pprh}
\setlength{\pdfpagewidth}{\pprw}
\setlength{\pdfpageheight}{\pprh}

% Make sure hyperref comes last of your loaded packages,
% to give it a fighting chance of not being over-written,
% since its job is to redefine many LaTeX commands.
\usepackage[pdftex]{hyperref}
\hypersetup{
pdftitle={SIGCHI Conference Proceedings Format},
pdfauthor={LaTeX},
pdfkeywords={SIGCHI, proceedings, archival format},
bookmarksnumbered,
pdfstartview={FitH},
colorlinks,
citecolor=black,
filecolor=black,
linkcolor=black,
urlcolor=black,
breaklinks=true,
}

% create a shortcut to typeset table headings
\newcommand\tabhead[1]{\small\textbf{#1}}


% End of preamble. Here it comes the document.
\begin{document}

%\title{Skimmable Video Digests for Informational Talks}
\title{Evaluating PeerDB}

\numberofauthors{1}
\author{
   \alignauthor Author Names\\
    \affaddr{University of California, Berkeley}\\
    \email{\{emails\}@cs.berkeley.edu}
}

\maketitle

\begin{abstract}
Simple document-oriented databases like MongoDB forgo many traditional features of relational databases in favor of better performance. 
This means that applications often need to build additional features on top of the simple provided features. 
For example, applications often have to resolve relationships between stored documents. 
Recursive queries incurred by resolving relationships introduce delays. 
To mitigate such delays, we developed PeerDB, a modification to MongoDB. 
PeerDB optimizes for the common use case where the main document only requires a subset of fields from any related documents. 
PeerDB stores subsets of related fields as subdocuments of the main document. 
In theory, this modification should reduce existing recursive read delays as applications will no longer need to recursively query documents in many cases. 
However, we have not quantitatively evaluated PeerDB to find out whether it improves performance of MongoDB. In addition, there is no method for automatically selecting which fields to embed using PeerDB.

We contribute a comparison of read and write times between three system: PeerDB, MongoDB and PostgreSQL. 
In addition, we make each comparison with both low-level queries and high-level web applications. 
We find that although PeerDB contributes to faster reads than other systems under some conditions, this performance gain comes with trade-offs (e.g. write time and traffic). 
To help developers manage these trade-offs when choosing which fields to embed, we propose and evaluate a new algorithm for automatically selecting which fields to embed based on a given read and write workload. 
Our algorithm compares many possible configurations using a novel cost model to predict expected impact, then returns the lowest cost configuration.
We test our algorithm under different workloads and parameter settings.
\end{abstract}

\vspace{-0.08in}
\keywords{
	MongoDB, PeerDB, Document stores
}

\vspace{-0.08in}


\section{Related Work}

Our algorithm embeds some document properties as sub documents in order to make it easier to reference whatever, our work is related to existing research in database cracking, materialized views, customizing key value stores, and discussions in the MongoDB development community.

\subsection{Materialized views}
A {\em view} is a function from a set of base tables to a derived table.
A {\em materialized view} is where you store tuples of a view in the database.
This way, database accesses to the materialized view can be faster than recomputing the view, especially when computing the view is expensive~\cite{Gupta1995}.
Materialized views for expensive queries would not work very well if you needed to recompute the materialized view whenever records were added or deleted, so past work has addressed how to efficiently maintain materialized views with incremental updates~\cite{Larson1985,Blakeley1986,Gupta1995,Zhou2007,Zhou2007a}.
Other work has provided methods for automatically selecting materialized views for a given workload in cases with lots of data~\cite{Agrawal2000,Yang1997}. 

Materialized views relate to our work because they create and update copies of data to decrease response time for expensive queries.
However, materialized views decrease database processing time on subsequent queries whereas we are interested in restructuring our data to decrease the number of round trips to the database. 

\subsection{Column store architecture and database cracking}
As we are restructuring our data to optimize read time, column store databases and database cracking are related to our project.
A {\em column store} relational database stores the values for each attribute contiguously, unlike traditional row store databases that store attributes of a record contiguously~\cite{Stonebraker}.
Such column store databases allow the DBMS to read only the values of the columns required for processing a given query so that they perform better in read-mostly applications.
Products such as Synbase IQ and KDB have demonstrated that this architecture improves performance~\cite{Stonebraker,French1995}. 

Idreos et al.~provide an optimization for column stores called {\em database cracking}~\cite{Pirk2007}.
In database cracking, a column $A$ is copied as $A_{CRK}$ when $A$ is first queried.
Then, $A_{CRK}$ is physically organized so that the values that satisfy the query are stored in contiguous space.
Thus, database cracking speeds up subsequent queries for similar values.
Follow up work proposed algorithms for updating cracked databases under high-volume insertions/deletions~\cite{Idreos2007}, and algorithms for increasing efficiency of tuple reconstruction for multi-attribute queries~\cite{Idreos2009}.

We are inspired by these successful methods for copying and reorganizing data to speed up read-heavy applications.
However, we can not directly use these methods in our case as they are intended for relational databases and we plan to reorganize data for fast reads on a non-relational database. 

\subsection{Adding functionality to non-relational databases} 
NoSQL-style databases sacrifice ``one-size-fits-all'' functionality for speed~\cite{Strauch}.
Thus, programmers build extra functionality on top of the simple database to satisfy application-specific needs.
Many research papers detail systems that supplement NoSQL databases to support complex quereies, ACID properties, and SLAs~\cite{Decandia2007,Chang,Beaver2010,Baker}. 

To our knowledge, no existing academic work addresses how to decrease the number of database round trips required to resolve recursive object relations in document store databases such as MongoDB.
The MongoDB manual~\cite{MongoDB2014} and a NoSQL survey paper~\cite{Strauch} notes application categories that inform when to (A) embed related objects and when to (B) reference related objects.
However, many applications do not comfortably fit either category.
So, application developers wrote guides for denormalizing objects so that you can both embed and reference objects~\cite{Wanschik2010}.
Unfortunately, following these guides requires a lot of effort and introduces opportunity for error.
Further, no one has qunatified benefits or downsides of denormalizing objects in MongoDB.

Our work creates and evaluates an application that supports denormalizing objects in MongoDB.



% \bibliography{papers}



% Balancing columns in a ref list is a bit of a pain because you
% either use a hack like flushend or balance, or manually insert
% a column break.  http://www.tex.ac.uk/cgi-bin/texfaq2html?label=balance
% multicols doesn't work because we're already in two-column mode,
% and flushend isn't awesome, so I choose balance.  See this
% for more info: http://cs.brown.edu/system/software/latex/doc/balance.pdf
%
% Note that in a perfect world balance wants to be in the first
% column of the last page.
%
% If balance doesn't work for you, you can remove that and
% hard-code a column break into the bbl file right before you
% submit:
%
% http://stackoverflow.com/questions/2149854/how-to-manually-equalize-columns-
% in-an-ieee-paper-if-using-bibtex
%
% Or, just remove \balance and give up on balancing the last page.
%
\balance

\bibliographystyle{acm-sigchi}
\bibliography{papers}
\end{document}
